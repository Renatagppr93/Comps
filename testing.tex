% Options for packages loaded elsewhere
\PassOptionsToPackage{unicode}{hyperref}
\PassOptionsToPackage{hyphens}{url}
%
\documentclass[
  english,
  man]{apa6}
\usepackage{amsmath,amssymb}
\usepackage{lmodern}
\usepackage{ifxetex,ifluatex}
\ifnum 0\ifxetex 1\fi\ifluatex 1\fi=0 % if pdftex
  \usepackage[T1]{fontenc}
  \usepackage[utf8]{inputenc}
  \usepackage{textcomp} % provide euro and other symbols
\else % if luatex or xetex
  \usepackage{unicode-math}
  \defaultfontfeatures{Scale=MatchLowercase}
  \defaultfontfeatures[\rmfamily]{Ligatures=TeX,Scale=1}
\fi
% Use upquote if available, for straight quotes in verbatim environments
\IfFileExists{upquote.sty}{\usepackage{upquote}}{}
\IfFileExists{microtype.sty}{% use microtype if available
  \usepackage[]{microtype}
  \UseMicrotypeSet[protrusion]{basicmath} % disable protrusion for tt fonts
}{}
\makeatletter
\@ifundefined{KOMAClassName}{% if non-KOMA class
  \IfFileExists{parskip.sty}{%
    \usepackage{parskip}
  }{% else
    \setlength{\parindent}{0pt}
    \setlength{\parskip}{6pt plus 2pt minus 1pt}}
}{% if KOMA class
  \KOMAoptions{parskip=half}}
\makeatother
\usepackage{xcolor}
\IfFileExists{xurl.sty}{\usepackage{xurl}}{} % add URL line breaks if available
\IfFileExists{bookmark.sty}{\usepackage{bookmark}}{\usepackage{hyperref}}
\hypersetup{
  pdftitle={The title},
  pdfauthor={First Author1 \& Ernst-August Doelle1,2},
  pdflang={en-EN},
  pdfkeywords={keywords},
  hidelinks,
  pdfcreator={LaTeX via pandoc}}
\urlstyle{same} % disable monospaced font for URLs
\usepackage{graphicx}
\makeatletter
\def\maxwidth{\ifdim\Gin@nat@width>\linewidth\linewidth\else\Gin@nat@width\fi}
\def\maxheight{\ifdim\Gin@nat@height>\textheight\textheight\else\Gin@nat@height\fi}
\makeatother
% Scale images if necessary, so that they will not overflow the page
% margins by default, and it is still possible to overwrite the defaults
% using explicit options in \includegraphics[width, height, ...]{}
\setkeys{Gin}{width=\maxwidth,height=\maxheight,keepaspectratio}
% Set default figure placement to htbp
\makeatletter
\def\fps@figure{htbp}
\makeatother
\setlength{\emergencystretch}{3em} % prevent overfull lines
\providecommand{\tightlist}{%
  \setlength{\itemsep}{0pt}\setlength{\parskip}{0pt}}
\setcounter{secnumdepth}{-\maxdimen} % remove section numbering
% Make \paragraph and \subparagraph free-standing
\ifx\paragraph\undefined\else
  \let\oldparagraph\paragraph
  \renewcommand{\paragraph}[1]{\oldparagraph{#1}\mbox{}}
\fi
\ifx\subparagraph\undefined\else
  \let\oldsubparagraph\subparagraph
  \renewcommand{\subparagraph}[1]{\oldsubparagraph{#1}\mbox{}}
\fi
% Manuscript styling
\usepackage{upgreek}
\captionsetup{font=singlespacing,justification=justified}

% Table formatting
\usepackage{longtable}
\usepackage{lscape}
% \usepackage[counterclockwise]{rotating}   % Landscape page setup for large tables
\usepackage{multirow}		% Table styling
\usepackage{tabularx}		% Control Column width
\usepackage[flushleft]{threeparttable}	% Allows for three part tables with a specified notes section
\usepackage{threeparttablex}            % Lets threeparttable work with longtable

% Create new environments so endfloat can handle them
% \newenvironment{ltable}
%   {\begin{landscape}\begin{center}\begin{threeparttable}}
%   {\end{threeparttable}\end{center}\end{landscape}}
\newenvironment{lltable}{\begin{landscape}\begin{center}\begin{ThreePartTable}}{\end{ThreePartTable}\end{center}\end{landscape}}

% Enables adjusting longtable caption width to table width
% Solution found at http://golatex.de/longtable-mit-caption-so-breit-wie-die-tabelle-t15767.html
\makeatletter
\newcommand\LastLTentrywidth{1em}
\newlength\longtablewidth
\setlength{\longtablewidth}{1in}
\newcommand{\getlongtablewidth}{\begingroup \ifcsname LT@\roman{LT@tables}\endcsname \global\longtablewidth=0pt \renewcommand{\LT@entry}[2]{\global\advance\longtablewidth by ##2\relax\gdef\LastLTentrywidth{##2}}\@nameuse{LT@\roman{LT@tables}} \fi \endgroup}

% \setlength{\parindent}{0.5in}
% \setlength{\parskip}{0pt plus 0pt minus 0pt}

% \usepackage{etoolbox}
\makeatletter
\patchcmd{\HyOrg@maketitle}
  {\section{\normalfont\normalsize\abstractname}}
  {\section*{\normalfont\normalsize\abstractname}}
  {}{\typeout{Failed to patch abstract.}}
\patchcmd{\HyOrg@maketitle}
  {\section{\protect\normalfont{\@title}}}
  {\section*{\protect\normalfont{\@title}}}
  {}{\typeout{Failed to patch title.}}
\makeatother
\shorttitle{Title}
\keywords{keywords\newline\indent Word count: X}
\DeclareDelayedFloatFlavor{ThreePartTable}{table}
\DeclareDelayedFloatFlavor{lltable}{table}
\DeclareDelayedFloatFlavor*{longtable}{table}
\makeatletter
\renewcommand{\efloat@iwrite}[1]{\immediate\expandafter\protected@write\csname efloat@post#1\endcsname{}}
\makeatother
\usepackage{lineno}

\linenumbers
\usepackage{csquotes}
\ifxetex
  % Load polyglossia as late as possible: uses bidi with RTL langages (e.g. Hebrew, Arabic)
  \usepackage{polyglossia}
  \setmainlanguage[]{english}
\else
  \usepackage[main=english]{babel}
% get rid of language-specific shorthands (see #6817):
\let\LanguageShortHands\languageshorthands
\def\languageshorthands#1{}
\fi
\ifluatex
  \usepackage{selnolig}  % disable illegal ligatures
\fi
\newlength{\cslhangindent}
\setlength{\cslhangindent}{1.5em}
\newlength{\csllabelwidth}
\setlength{\csllabelwidth}{3em}
\newenvironment{CSLReferences}[2] % #1 hanging-ident, #2 entry spacing
 {% don't indent paragraphs
  \setlength{\parindent}{0pt}
  % turn on hanging indent if param 1 is 1
  \ifodd #1 \everypar{\setlength{\hangindent}{\cslhangindent}}\ignorespaces\fi
  % set entry spacing
  \ifnum #2 > 0
  \setlength{\parskip}{#2\baselineskip}
  \fi
 }%
 {}
\usepackage{calc}
\newcommand{\CSLBlock}[1]{#1\hfill\break}
\newcommand{\CSLLeftMargin}[1]{\parbox[t]{\csllabelwidth}{#1}}
\newcommand{\CSLRightInline}[1]{\parbox[t]{\linewidth - \csllabelwidth}{#1}\break}
\newcommand{\CSLIndent}[1]{\hspace{\cslhangindent}#1}

\title{The title}
\author{First Author\textsuperscript{1} \& Ernst-August Doelle\textsuperscript{1,2}}
\date{}


\authornote{

Add complete departmental affiliations for each author here. Each new line herein must be indented, like this line.

Enter author note here.

The authors made the following contributions. First Author: Conceptualization, Writing - Original Draft Preparation, Writing - Review \& Editing; Ernst-August Doelle: Writing - Review \& Editing.

Correspondence concerning this article should be addressed to First Author, Postal address. E-mail: \href{mailto:my@email.com}{\nolinkurl{my@email.com}}

}

\affiliation{\vspace{0.5cm}\textsuperscript{1} Wilhelm-Wundt-University\\\textsuperscript{2} Konstanz Business School}

\abstract{
One or two sentences providing a \textbf{basic introduction} to the field, comprehensible to a scientist in any discipline.

Two to three sentences of \textbf{more detailed background}, comprehensible to scientists in related disciplines.

One sentence clearly stating the \textbf{general problem} being addressed by this particular study.

One sentence summarizing the main result (with the words ``\textbf{here we show}'' or their equivalent).

Two or three sentences explaining what the \textbf{main result} reveals in direct comparison to what was thought to be the case previously, or how the main result adds to previous knowledge.

One or two sentences to put the results into a more \textbf{general context}.

Two or three sentences to provide a \textbf{broader perspective}, readily comprehensible to a scientist in any discipline.
}



\begin{document}
\maketitle

\hypertarget{methods}{%
\section{Methods}\label{methods}}

We report how we determined our sample size, all data exclusions (if any), all manipulations, and all measures in the study.

\hypertarget{participants}{%
\subsection{Participants}\label{participants}}

\hypertarget{material}{%
\subsection{Material}\label{material}}

\hypertarget{procedure}{%
\subsection{Procedure}\label{procedure}}

\hypertarget{data-analysis}{%
\subsection{Data analysis}\label{data-analysis}}

We used R {[}Version 4.0.3; R Core Team (2021){]} and the R-package \emph{papaja} {[}Version 0.1.0.9997; Aust and Barth (2020){]} for all our analyses.

\hypertarget{results}{%
\section{Results}\label{results}}

\hypertarget{discussion}{%
\section{Discussion}\label{discussion}}

\newpage

\hypertarget{references}{%
\section{References}\label{references}}

\begingroup
\setlength{\parindent}{-0.5in}
\setlength{\leftskip}{0.5in}

\hypertarget{refs}{}
\begin{CSLReferences}{1}{0}
\leavevmode\hypertarget{ref-aban_relationship_2019}{}%
Aban, C. J. I., Perez, V. E. B., Ricarte, K. K. G., \& Chiu, J. L. (2019). The relationship of organizational commitment, job satisfaction, and perceived organizational support of telecommuters in the national capital region. \emph{Review of Integrative Business and Economics Research}, \emph{8}, 162--197.

\leavevmode\hypertarget{ref-allen_how_2015}{}%
Allen, T. D., Golden, T. D., \& Shockley, K. M. (2015). How effective is telecommuting? Assessing the status of our scientific findings. \emph{Psychological Science in the Public Interest}, \emph{16}(2), 40--68.

\leavevmode\hypertarget{ref-R-papaja}{}%
Aust, F., \& Barth, M. (2020). \emph{{papaja}: {Create} {APA} manuscripts with {R Markdown}}. Retrieved from \url{https://github.com/crsh/papaja}

\leavevmode\hypertarget{ref-bailey_review_2002}{}%
Bailey, D. E., \& Kurland, N. B. (2002). A review of telework research: Findings, new directions, and lessons for the study of modern work. \emph{Journal of Organizational Behavior: The International Journal of Industrial, Occupational and Organizational Psychology and Behavior}, \emph{23}(4), 383--400.

\leavevmode\hypertarget{ref-bakker_job_2017}{}%
Bakker, A. B., \& Demerouti, E. (2017). Job demands--resources theory: Taking stock and looking forward. \emph{Journal of Occupational Health Psychology}, \emph{22}(3), 273.

\leavevmode\hypertarget{ref-baruch_teleworking_2000}{}%
Baruch, Y. (2000). Teleworking: Benefits and pitfalls as perceived by professionals and managers. \emph{New Technology, Work and Employment}, \emph{15}(1), 34--49.

\leavevmode\hypertarget{ref-beckers_worktime_2012}{}%
Beckers, D. G., Kompier, M. A., Kecklund, G., \& Härmä, M. (2012). Worktime control: Theoretical conceptualization, current empirical knowledge, and research agenda. \emph{Scandinavian Journal of Work, Environment \& Health}, 291--297.

\leavevmode\hypertarget{ref-beckmann_self-managed_2017}{}%
Beckmann, M., Cornelissen, T., \& Kräkel, M. (2017). Self-managed working time and employee effort: Theory and evidence. \emph{Journal of Economic Behavior \& Organization}, \emph{133}, 285--302.

\leavevmode\hypertarget{ref-biron_when_2016}{}%
Biron, M., \& Veldhoven, M. van. (2016a). When control becomes a liability rather than an asset: Comparing home and office days among part-time teleworkers. \emph{Journal of Organizational Behavior}, \emph{37}(8), 1317--1337.

\leavevmode\hypertarget{ref-biron_when_2016-1}{}%
Biron, M., \& Veldhoven, M. van. (2016b). When control becomes a liability rather than an asset: Comparing home and office days among part-time teleworkers. \emph{Journal of Organizational Behavior}, \emph{37}(8), 1317--1337.

\leavevmode\hypertarget{ref-bloom_does_2015}{}%
Bloom, N., Liang, J., Roberts, J., \& Ying, Z. J. (2015a). Does working from home work? Evidence from a chinese experiment. \emph{The Quarterly Journal of Economics}, \emph{130}(1), 165--218.

\leavevmode\hypertarget{ref-bloom_does_2015-1}{}%
Bloom, N., Liang, J., Roberts, J., \& Ying, Z. J. (2015b). Does working from home work? Evidence from a chinese experiment. \emph{The Quarterly Journal of Economics}, \emph{130}(1), 165--218.

\leavevmode\hypertarget{ref-capitano_when_2018}{}%
Capitano, J., \& Greenhaus, J. H. (2018). When work enters the home: Antecedents of role boundary permeability behavior. \emph{Journal of Vocational Behavior}, \emph{109}, 87--100.

\leavevmode\hypertarget{ref-charalampous_systematically_2019}{}%
Charalampous, M., Grant, C. A., Tramontano, C., \& Michailidis, E. (2019). Systematically reviewing remote e-workers' well-being at work: A multidimensional approach. \emph{European Journal of Work and Organizational Psychology}, \emph{28}(1), 51--73.

\leavevmode\hypertarget{ref-delanoeije_boundary_2019}{}%
Delanoeije, J., Verbruggen, M., \& Germeys, L. (2019a). Boundary role transitions: A day-to-day approach to explain the effects of home-based telework on work-to-home conflict and home-to-work conflict. \emph{Human Relations}, \emph{72}(12), 1843--1868.

\leavevmode\hypertarget{ref-delanoeije_boundary_2019-1}{}%
Delanoeije, J., Verbruggen, M., \& Germeys, L. (2019b). Boundary role transitions: A day-to-day approach to explain the effects of home-based telework on work-to-home conflict and home-to-work conflict. \emph{Human Relations}, \emph{72}(12), 1843--1868.

\leavevmode\hypertarget{ref-derks_smartphone_2015}{}%
Derks, D., Duin, D. van, Tims, M., \& Bakker, A. B. (2015). Smartphone use and work--home interference: The moderating role of social norms and employee work engagement. \emph{Journal of Occupational and Organizational Psychology}, \emph{88}(1), 155--177.

\leavevmode\hypertarget{ref-dutcher_effects_2012}{}%
Dutcher, E. G. (2012). The effects of telecommuting on productivity: An experimental examination. The role of dull and creative tasks. \emph{Journal of Economic Behavior \& Organization}, \emph{84}(1), 355--363.

\leavevmode\hypertarget{ref-fay_coworker_2011}{}%
Fay, M. J., \& Kline, S. L. (2011a). Coworker relationships and informal communication in high-intensity telecommuting. \emph{Journal of Applied Communication Research}, \emph{39}(2), 144--163.

\leavevmode\hypertarget{ref-fay_coworker_2011-1}{}%
Fay, M. J., \& Kline, S. L. (2011b). Coworker relationships and informal communication in high-intensity telecommuting. \emph{Journal of Applied Communication Research}, \emph{39}(2), 144--163.

\leavevmode\hypertarget{ref-fonner_why_2010}{}%
Fonner, K. L., \& Roloff, M. E. (2010). Why teleworkers are more satisfied with their jobs than are office-based workers: When less contact is beneficial. \emph{Journal of Applied Communication Research}, \emph{38}(4), 336--361.

\leavevmode\hypertarget{ref-fonner_all_2012}{}%
Fonner, K. L., \& Stache, L. C. (2012a). All in a day's work, at home: Teleworkers' management of micro role transitions and the work--home boundary. \emph{New Technology, Work and Employment}, \emph{27}(3), 242--257.

\leavevmode\hypertarget{ref-fonner_teleworkers_2012}{}%
Fonner, K. L., \& Stache, L. C. (2012b). Teleworkers' boundary management: Temporal, spatial, and expectation-setting strategies. In \emph{Virtual work and human interaction research} (pp. 31--58). {IGI} Global.

\leavevmode\hypertarget{ref-freud_letter_1951}{}%
Freud, S. (1951). A letter from {Freud}. \emph{The American Journal of Psychiatry}.

\leavevmode\hypertarget{ref-gajendran_good_2007}{}%
Gajendran, R. S., \& Harrison, D. A. (2007). The good, the bad, and the unknown about telecommuting: Meta-analysis of psychological mediators and individual consequences. \emph{Journal of Applied Psychology}, \emph{92}(6), 1524.

\leavevmode\hypertarget{ref-gajendran_are_2015}{}%
Gajendran, R. S., Harrison, D. A., \& Delaney-Klinger, K. (2015). Are telecommuters remotely good citizens? Unpacking telecommuting's effects on performance via i-deals and job resources. \emph{Personnel Psychology}, \emph{68}(2), 353--393.

\leavevmode\hypertarget{ref-golden_applying_2009}{}%
Golden, T. D. (2009). Applying technology to work: Toward a better understanding of telework. \emph{Organization Management Journal}, \emph{6}(4), 241--250.

\leavevmode\hypertarget{ref-golden_unpacking_2019}{}%
Golden, T. D., \& Gajendran, R. S. (2019). Unpacking the role of a telecommuter's job in their performance: Examining job complexity, problem solving, interdependence, and social support. \emph{Journal of Business and Psychology}, \emph{34}(1), 55--69.

\leavevmode\hypertarget{ref-golden_impact_2005}{}%
Golden, T. D., \& Veiga, J. F. (2005). The impact of extent of telecommuting on job satisfaction: Resolving inconsistent findings. \emph{Journal of Management}, \emph{31}(2), 301--318.

\leavevmode\hypertarget{ref-golden_telecommutings_2006}{}%
Golden, T. D., Veiga, J. F., \& Simsek, Z. (2006a). Telecommuting's differential impact on work-family conflict: Is there no place like home? \emph{Journal of Applied Psychology}, \emph{91}(6), 1340.

\leavevmode\hypertarget{ref-golden_telecommutings_2006-1}{}%
Golden, T. D., Veiga, J. F., \& Simsek, Z. (2006b). Telecommuting's differential impact on work-family conflict: Is there no place like home? \emph{Journal of Applied Psychology}, \emph{91}(6), 1340.

\leavevmode\hypertarget{ref-igeltjorn_homebased_2020}{}%
Igeltjørn, A., \& Habib, L. (2020). Homebased telework as a tool for inclusion? A literature review of telework, disabilities and work-life balance. In \emph{International conference on human-computer interaction} (pp. 420--436). Springer.

\leavevmode\hypertarget{ref-lee_telework_2007}{}%
Lee, H., Shin, B., \& Higa, K. (2007). Telework vs. Central work: A comparative view of knowledge accessibility. \emph{Decision Support Systems}, \emph{43}(3), 687--700.

\leavevmode\hypertarget{ref-martin_is_2012}{}%
Martin, B. H., \& MacDonnell, R. (2012). Is telework effective for organizations? \emph{Management Research Review}.

\leavevmode\hypertarget{ref-maruyama_multivariate_2009}{}%
Maruyama, T., Hopkinson, P. G., \& James, P. W. (2009). A multivariate analysis of work--life balance outcomes from a large-scale telework programme. \emph{New Technology, Work and Employment}, \emph{24}(1), 76--88.

\leavevmode\hypertarget{ref-mcdonald_social_2015}{}%
McDonald, R. I., \& Crandall, C. S. (2015). Social norms and social influence. \emph{Current Opinion in Behavioral Sciences}, \emph{3}, 147--151.

\leavevmode\hypertarget{ref-morris_normology_2015}{}%
Morris, M. W., Hong, Y., Chiu, C., \& Liu, Z. (2015). Normology: Integrating insights about social norms to understand cultural dynamics. \emph{Organizational Behavior and Human Decision Processes}, \emph{129}, 1--13.

\leavevmode\hypertarget{ref-moser_role_2013}{}%
Moser, K. S., \& Axtell, C. M. (2013). The role of norms in virtual work: A review and agenda for future research. \emph{Journal of Personnel Psychology}, \emph{12}(1), 1.

\leavevmode\hypertarget{ref-park_employee_2019}{}%
Park, S., \& Park, S. (2019). Employee adaptive performance and its antecedents: Review and synthesis. \emph{Human Resource Development Review}, \emph{18}(3), 294--324.

\leavevmode\hypertarget{ref-purvanova_face--face_2014}{}%
Purvanova, R. K. (2014). Face-to-face versus virtual teams: What have we really learned? \emph{The Psychologist-Manager Journal}, \emph{17}(1), 2.

\leavevmode\hypertarget{ref-R-base}{}%
R Core Team. (2021). \emph{R: A language and environment for statistical computing}. Vienna, Austria: R Foundation for Statistical Computing. Retrieved from \url{https://www.R-project.org/}

\leavevmode\hypertarget{ref-raghuram_virtual_2019}{}%
Raghuram, S., Hill, N. S., Gibbs, J. L., \& Maruping, L. M. (2019). Virtual work: Bridging research clusters. \emph{Academy of Management Annals}, \emph{13}(1), 308--341.

\leavevmode\hypertarget{ref-steward_changing_2000}{}%
Steward, B. (2000). Changing times: The meaning, measurement and use of time in teleworking. \emph{Time \& Society}, \emph{9}(1), 57--74.

\leavevmode\hypertarget{ref-stutzer_role_2004}{}%
Stutzer, A., \& Lalive, R. (2004). The role of social work norms in job searching and subjective well-being. \emph{Journal of the European Economic Association}, \emph{2}(4), 696--719.

\end{CSLReferences}

\endgroup


\end{document}
